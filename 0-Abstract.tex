\chapter*{Abstract}
\label{chap:Abstract}
\addcontentsline{toc}{chapter}{Abstract}


The aim of this work is to investigate the transport properties of different silicon (Si)-based heterostructures. On the one hand the TCO/a-Si:H/c-Si layer stack as charge carrier selective contact of Si heterojunction (SHJ) solar cells, consisting of a transparent, conductive oxide (TCO), a hydrogenated amorphous Si (a-Si:H) layer and the crystalline Si (c-Si) absorber. On the other hand, a Si($p^+$/$n^+$) layer stack made of poly-crystalline Si (poly-Si) as a tunneling junction for interconnecting a Si bottom and a perovskite top cell in monolithic, two-terminal perovskite/Si tandem solar cells. 

SHJ solar cells are among the most efficient single-junction devices and hold the current world record for Si single-junction solar cells. A drawback, however, are the comparatively high transport losses in the charge carrier-selective contacts, which are analyzed and reduced in this work. 

Using detailed/designated test structures, this work shows that the high transport losses are mainly determined by the TCO/a-Si:H and the a-Si:H/c-Si interface. For the TCO, low oxygen concentrations and thus high charge carrier concentrations are advantageous in terms of charge carrier transport, both for holes and for electrons. Optical and electrical properties can be effectively decoupled with a layer stack of oxygen-rich bulk material and oxygen-poor interface layers. 

It is shown here that the porous a-Si:H($i$1) interfacial layer, which is used to prevent epitaxial growth and thus to provide excellent surface passivation, is mainly responsible for the high transport losses at the a-Si:H/c-Si interface. 
An important finding is that this effect occurs for both electron and hole contact and is therefore probably due to the poor conductivity of such a porous a-Si:H($i$) layer or possibly due to poor surface coverage and thus an effectively small contact area.
Various possibilities are presented for the hole contact, which significantly reduce the contact resistance and thus also the series resistance of SHJ solar cells, which is reflected in increased efficiency. 



Tandem solar cells are the logical successors of Si single-junction solar cells in order to achieve even higher levels of efficiency and thus even lower levelized costs of electricity.
It is important for the industrial production of such tandem solar cells to use synergy effects by using processes that are already established in the industry. 

In this work a Si bottom cell is developed (TOPerc), which is based on the industrial mainstream passivated emitter and rear cell (PERC) technology and uses a tunnel oxide passivating contact (TOPCon) at the front and therefore allows for significantly higher open-circuit voltages ($V_{oc} \ge 690 ~\textrm{mV}$) compared to the conventional diffused emitter.
TOPerc is therefore an upgrade for the PERC technology to make it ready for industrial perovskite/Si tandem solar cells with just a few additional process steps. 

Furthermore, a TOPCon-based poly-Si tunneling junction with an extremely steep $p$/$n$-junction is developed, which is compatible with the TOPerc cell concept.
The influence of the post-deposition thermal budget on the diffusion of the dopants, the crystallization and the associated effects on the charge carrier transport in this poly-Si tunneling junction is investigated. 
The parasitic interdiffusion of dopants at the $p^+$/$n^+$ interface and the associated reduction in the doping level on both sides of the tunneling junction and the widening of the depletion zone close the interface is identified as the most limiting factor.
Therefore, two approaches are presented to prevent this interdiffusion: The alloying of the poly-Si layers with carbon and the use of an ultra-thin diffusion-blocking interlayer.
In the context of this work Si tunneling junctions are fabricated, which have very low contact resistivity of $\rho_{c} \approx 10 ~\textrm{m}\upOmega\textrm{cm}^2$ for the entire poly-Si ($p^+$/$n^+$)/SiO$_x$/c-Si layer stack and high implied open-circuit voltages i$V_{oc}$ above $720 ~\textrm{mV}$.
Even with a very lean, industrially preferable process sequence with only one rapid thermal process, very good results are obtained ($\rho_{c} \approx 30 ~\textrm{m}\upOmega\textrm{cm}^2$, i$V_{oc} > 710 ~\textrm{mV}$). 

\clearpage